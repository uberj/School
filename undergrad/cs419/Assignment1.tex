\documentclass[12pt]{article}
\usepackage[hmargin=1in, vmargin=1in]{geometry}
\usepackage{fancyhdr}
\usepackage{setspace}
\pagestyle{fancy}
\usepackage[small]{caption}
\usepackage{lastpage}
\usepackage{graphicx}
\usepackage{verbatim}
\DeclareGraphicsExtensions{.png}
\usepackage{url}

\def\author{Jacques Uber}
\def\title{Django: https://www.djangoproject.com/}
\def\date{\today}

\fancyhf{} % clear all header and footer fields
\fancyhead[LO]{\author}
\fancyhead[RO]{\date}
\renewcommand{\headrulewidth}{0pt}
% The weird spacing here is to get the spacing of \thepage to be right.
\fancyfoot[C]{\thepage\
                    / 2}

\setcounter{secnumdepth}{0}
\setlength{\parindent}{0pt}
\setlength{\parskip}{4mm}
\linespread{1}
\begin{document}
\begin{center}
\underline{
\large{\title}
}
\end{center}
\doublespacing

\section{Project Membership}
Django has 4 BDFLs. A BDFL is a Benevolent Dictators for Life. The BDFLs are
seen as the 'community' leaders and get to make the hard decisions when the
community can't agree on something. According to Django's website, there are
currently 42 core developers (developers with commit access). On github, it
shows that Django has had over 100 people submit patches. Django developers
communicate on a mailing list that is different from the mailing list used by
django users use. There is also an official irc channel, \#django, on
irc.freenode.net where django users and developers hang out and chat.

\section{Project Infrastructure}
Django has a split infrastructure. Both the Django source code, wiki, and issue
tracker used to all be in the same place: TracSubversion. Also, SVN used to be
the VCS. Eventually, the community decided that they wanted to switch to git
and use github.com to host their source code reposity. Django now has their
wiki and issue tracker under the original djangoproject.com domain and their
source code and pull requests on github.com.

Django has both their user and development mailing lists indexed by google
groups.

To maintain a log of changes to Django, a web blog is kept at
https://www.djangoproject.com/weblog/. This is a great resource for starting to
understand how the community works and the changes that have been made to
Django over the years. You can easily go back to July 2005 and read their first
post 'Preparing for launch'.

\section{Django TL;DR}
Django was started at Lawrence Journal-World as a tool to develop web
applications more easily. It had four original contributers. Django is a web
framework, which means that it isn't an application itself, but is instead a
framwork for building applications. Django involves a lot of moving parts
including but not limited to: an ORM (Object Relational Mapper), cacheing
subsystem, form library, wsgi wrapper, and templating language. Django prides
itself as being "The Web framework for perfectionists with deadlines".

\section{Why Django?}
I have been writing a lot of code destined to support websites for my job
lately. This code has almost exclusivly been done using the Django
framwork. I know my way around the source code already and have been in
their IRC channel on freenode to ask questions about development with django
(never about developing Django itself). I have also been on the Django
developer mailing list for nearly 8 months and I try to read the daily digest
every night (I don't always have time). I think that Django would be a good
project to give back to. On their issue tracker they have a '"Easy Pickings"
Tickets' section that lists tickets that I could help with.

\section{Relevant Links}
\begin{itemize}
\item Django's ticket tracker: https://code.djangoproject.com/wiki/Reports
\item Django's source code: https://github.com/django/django
\item Django's community page: https://www.djangoproject.com/community/
\end{itemize}

\end{document}
