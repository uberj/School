\documentclass[12pt]{article}
\usepackage[hmargin=1in, vmargin=1in]{geometry}
\usepackage{fancyhdr}
\usepackage{setspace}
\pagestyle{fancy}
\usepackage[small]{caption}
\usepackage{lastpage}
\usepackage{listings}
\usepackage{verbatim}
\usepackage{url}

% Custom colors
\usepackage{color}
\definecolor{deepblue}{rgb}{0,0,0.5}
\definecolor{deepred}{rgb}{0.6,0,0}
\definecolor{deepgreen}{rgb}{0,0.5,0}
\definecolor{grey}{gray}{0.9}


\lstset{
    tabsize=2,
    breaklines=true,
    showstringspaces=false
    backgroundcolor=\color{grey},
    basicstyle=\ttfamily\scriptsize,
}


\def\author{Jacques Uber}
\def\title{Assignment 2: Contributing to Django}
\def\date{\today}

\fancyhf{} % clear all header and footer fields
\fancyhead[LO]{\author}
\fancyhead[RO]{\date}
\renewcommand{\headrulewidth}{0pt}
% The weird spacing here is to get the spacing of \thepage to be right.
\fancyfoot[C]{\thepage\
                    / 2}

\setcounter{secnumdepth}{0}
\setlength{\parindent}{0pt}
\setlength{\parskip}{4mm}
\linespread{1}
% Talk about how the hinge works and where the power switch it relative to how the hinge swings.
% Pictures:
%   Use copyright: Name, Year
% Get rid of value language
%
% OO: Wednesday 9-12
%
\begin{document}
%\fancyhead[CO]{\title}
\begin{center}
\underline{
\large{\title}
}
\end{center}
\singlespacing

\subsection{Looking for broken tests}
I initially thought I was going to find a broken test and contribute a fix for
it. To search for a failing test I checked out the code base of django via git
and found it pretty easy to run the test suite. The process was documented
here: https://code.djangoproject.com/wiki/RunningDjangoTests. I gave up on this
method of contributing after I found that no tests were failing.

\subsection{Easy Pickings}
In my initial write-up I mentioned that Django's ticketing system (TRAC) had a
section of 'easy pickings' tickets. These tickets are supposed to be easy to
complete. After looking over this list I realized that most of them would not
be easy to work on. A lot of the tickets have been open for a while and take a
lot of background knowledge to understand.

\section{Finding a bug}
After my initial attempts to force myself to contribute to Django, I took a
break. At work I have been developing a webapp that uses
Django and in one of the project's modules I needed to use Django's manual
database transaction API.  I wrote my code using this API and things seemed to
work correctly. I then decided to test my code; this is where I found a bug...
eventually. I spend a few hours banging my head against the code, reading
documentation, and stepping through things with a debugger. I even tried
looking at the source code to see what was going on, but I couldn't find
anything! I came back a day later and realized what was happening in the code,
so I decided to get on the django IRC channel (\#django on freenode) and ask
if anyone could help me out. Here are the IRC logs:

\begin{lstlisting}
2013-04-24 03:32:17  uberj   I was playing with transactions in a test case
and I found something interesting which is _sort_ of documented. The docs on
TransactionTestCase say "It also prevents the code under test from issuing any
commit or rollback operations on the database..." which could be true if you
take the word 'prevents' to mean 'overrides functions and doesn't complain when
you use them'. For what I'm talking about see this:

2013-04-24 03:32:17  uberj
https://github.com/django/django/blob/master/django/test/testcases.py#L75 .
This bit super hard and I wasted time before I realized that I needed to use
TransactionTestCase and not TestCase. Why doen't noop throw an excpetion or at
least print 'WARNING: you probably want to be using TransactionTestCase'?

2013-04-24 03:34:42  willy1234x1   uberj: Fix it and submit a pull request?
2013-04-24 03:34:48  willy1234x1   You mean nop though not noop?
2013-04-24 03:34:54  uberj   yes
2013-04-24 03:34:56  uberj   sorry
2013-04-24 03:36:03  uberj   well, adding an exception might break some
peoples tests (I'm not sure how people will feel about that). I also don't know
how people feel about using 'print'; it could be considered bad style

2013-04-24 03:38:50  uberj   willy1234x1: on a more meta note, where *is*
the correct venue for a discussion on this? github pull request? SVN ticket?
here in IRC?
2013-04-24 03:39:01  uberj   *TRAC ticket

2013-04-24 03:53:34  luminous   uberj: post a ticket and then send to
mailing list to discuss?

2013-04-24 03:55:40  uberj   luminous: okay
\end{lstlisting}

So I filed a ticket: https://code.djangoproject.com/ticket/20316 . When I
filed the ticket I made sure to file it under the 'Testing' component. A core
developer named charettes responded very quickly to my ticket and was able to
classify my bug as a bug in the documentation and not in the code. The ticket
is now 'Accepted' and is filed under Documentation.

I never did post the the mailing list because I thought things were pretty well
explained in the ticket.

\section{What I learned}
I learned that the Django community is pretty open to people filing tickets. It
was also nice to be able to propose some ideas and have those ideas invalidated
in a nice way.

\end{document}
